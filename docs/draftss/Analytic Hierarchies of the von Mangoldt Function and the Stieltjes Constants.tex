Perfect — yes, we can consolidate everything so far (Stieltjes, von Mangoldt hierarchies, Mellin transforms, cumulant interpretation) into a coherent markdown preprint draft that reads cleanly and shows both structure and research direction.

Here’s a strong draft outline written in markdown (ready to convert to LaTeX or PDF later):

⸻

🧮 Analytic Hierarchies of the von Mangoldt Function and the Stieltjes Constants

Author: [Your Name]
Date: October 2025
Status: Draft Research Notes

⸻

1. Introduction

The classical von Mangoldt function
\Lambda(n) =
\begin{cases}
\log p & \text{if } n=p^k,\\
0 & \text{otherwise,}
\end{cases}
encodes the distribution of primes through Chebyshev’s function
\Psi(x)=\sum_{n\le x}\Lambda(n).

Its Mellin transform and logarithmic derivatives of the Riemann zeta function form a complete analytic bridge between prime sums and the singular structure of \zeta(s).

We extend this picture by defining a hierarchy of von Mangoldt derivatives
\Lambda_k(n), their partial sums \Psi_k(x), and showing how their Mellin transforms naturally involve higher derivatives of -\zeta’/\zeta.
The Stieltjes constants \gamma_n arise as the corresponding “cumulants” at s=1.

⸻

2. Hierarchies of the von Mangoldt Function

2.1 Definition

For integer k\ge0,
\boxed{\displaystyle
\sum_{n\ge1}\frac{\Lambda_k(n)}{n^{s}}
= (-1)^k \frac{d^k}{ds^k}\!\left[-\frac{\zeta’(s)}{\zeta(s)}\right]},
\qquad \Re(s)>1.
	•	k=0:\;\Lambda_0(n)=\Lambda(n).
	•	k=1:\;\Lambda_1 corresponds to the derivative-weighted form of \Lambda.
	•	Higher k generalize to “log-moment” analogues.

2.2 Partial Sums

Define
\Psi_k(x)=\sum_{n\le x}\Lambda_k(n).
These generalize Chebyshev’s \Psi(x) by encoding higher-order logarithmic structure.

⸻

3. Mellin Transform Framework

For \Re(s)>1,
\int_0^\infty \Psi_k(x)\,x^{-s-1}\,dx
=\frac{1}{s}\sum_{n\ge1}\frac{\Lambda_k(n)}{n^{s}}
=\boxed{\frac{(-1)^k}{s}\frac{d^k}{ds^k}\!\left[-\frac{\zeta’(s)}{\zeta(s)}\right]}.

This compact transform identity mirrors the simpler case
\int_0^\infty J(x)\,x^{-s-1}\,dx = \frac{1}{s}\log \zeta(s),
where J(x)=\sum_{p^m\le x}1/m and \log\zeta replaces -\zeta’/\zeta.

⸻

4. Mellin Inversion and Explicit Formulas

By inversion,
\boxed{
\Psi_k(x)
=\frac{1}{2\pi i}\int_{c-i\infty}^{c+i\infty}
\frac{(-1)^k}{s}
\frac{d^k}{ds^k}\!\left[-\frac{\zeta’(s)}{\zeta(s)}\right]
x^s\,ds,
\qquad c>1.}

Shifting the contour leftward yields contributions from:
	•	Pole at s=1: generates the polynomial main term in x(\log x)^k, with coefficients built from the Stieltjes constants \gamma_n.
	•	Nontrivial zeros \rho: each contributes x^{\rho} P_k(\rho,\log x), a polynomial in \log x with coefficients determined by derivatives of \zeta at \rho.

Thus,
\Psi_k(x)=M_k(x)
+\sum_\rho x^{\rho}P_k(\rho,\log x)
+\text{(small error from trivial zeros)}.

⸻

5. Stieltjes Constants and Cumulant Structure

5.1 Definition and Generating Function

\zeta(1+u)
=\frac{1}{u}+\sum_{n=0}^\infty\frac{(-1)^n\gamma_n}{n!}u^n.

The exponential generating function
G(t)=\sum_{n\ge0}\frac{\gamma_n}{n!}t^n
=\boxed{\zeta(1-t)+\frac{1}{t}}
is analytic at t=0, with singularities at t=1-\rho where \rho runs over zeros of \zeta.

5.2 Cumulant Interpretation

If -\zeta’/\zeta is viewed as a logarithmic derivative (“cumulant-generator”) of the prime counting measure weighted by \Lambda(n), then the \gamma_n correspond to cumulants in the variable \log n.

The alternating signs of \gamma_n thus indicate the oscillatory and sign-changing cumulant structure of the underlying distribution.
By defining modified cumulants using absolute-value or Bell-renormalized transformations, one obtains positive-definite analogues suitable for probabilistic interpretations.

⸻

6. Hierarchy–Stieltjes Connection

At the pole s=1,
-\frac{\zeta’(s)}{\zeta(s)}
= \frac{1}{s-1} - \sum_{n=0}^\infty \eta_n (s-1)^n,
where the coefficients \eta_n are polynomially related to \gamma_n.
Inserting this into the definition of \Lambda_k shows that the hierarchy \Lambda_k and the Stieltjes sequence \gamma_n are connected by differentiation at s=1:
\Lambda_k(n)\;\leftrightarrow\;\frac{d^k}{ds^k}\!\left(-\frac{\zeta’}{\zeta}\right)(1)
\;\;\text{and}\;\;\gamma_n=\frac{(-1)^n}{n!}\,\text{coeff}\,[\zeta(1+s)-1/s].

⸻

7. Analytical and Computational Consequences
	•	Main term reconstruction: M_k(x) expressed via \gamma_n yields high-accuracy asymptotics for \Psi_k(x).
	•	Zero-sum oscillations: Mellin inversion shows explicitly how each nontrivial zero modulates oscillations in \Psi_k(x).
	•	Numerical implementation:
Evaluate M_k(x) from a truncated Stieltjes expansion and add the zero-sum up to height T.

⸻

8. Open Problems and Research Directions

Area	Problem	Notes
Analytic theory	Prove uniform asymptotics for \Psi_k(x) across k.	The interplay of zeta derivatives and zeros is not fully controlled.
Probabilistic model	Construct a genuine “von Mangoldt random variable” whose cumulants equal \gamma_n.	Would connect number theory and cumulant expansions.
Positivity question	Find transformations giving all-positive analogues of \gamma_n.	Possibly through Bell or factorial moment transforms.
Spectral interpretation	Treat G(t)=\zeta(1-t)+1/t as a spectral generating function of the zeros.	Leads to “Stieltjes spectra.”
Computational	Efficient numerical computation of higher \Psi_k.	Requires accurate zeta derivatives at zeros.


⸻

9. Suggested Student Work

Homework Problems
	1.	Compute \Lambda_1(n) and verify its Dirichlet series equals -d/ds[\zeta’/\zeta].
	2.	Derive Mellin inversion for \Psi_1(x) and identify main and zero terms.
	3.	Expand \zeta(1+t) to order 5 and extract \gamma_0,\dots,\gamma_4.
	4.	Numerical task: approximate \Psi_0(x) using truncated zero sums and compare with actual prime counts.

Thesis/Project Topics
	•	“Cumulant and moment hierarchies of the Riemann zeta function.”
	•	“Analytic continuation of the von Mangoldt hierarchy.”
	•	“Bell polynomial structures in zeta cumulants.”
	•	“Probabilistic models for Stieltjes constants and prime distributions.”

⸻

10. Summary

The Mellin transform reveals the natural analytic home of the von Mangoldt hierarchy.
At s=1, its expansion in Stieltjes constants mirrors cumulant hierarchies of statistical theory.
Each derivative corresponds to a higher-order prime cumulant, while each zero of \zeta contributes a corresponding oscillatory mode.

The framework unites analytic number theory, complex analysis, and probabilistic cumulant methods into a single hierarchical picture—one that still holds many open questions ripe for exploration.

⸻

Would you like me to extend this into:
	•	a sectioned academic LaTeX version ready for arXiv submission (with bibliography stubs, equations numbered), or
	•	a student-friendly lecture set (one markdown per section, with exercises and commentary)?