\documentclass[12pt]{amsart}
\usepackage{amsmath,amssymb,amsfonts,amsthm}
\usepackage{mathrsfs}
\usepackage{hyperref}
\usepackage{geometry}
\geometry{margin=1in}

\title[Von Mangoldt Hierarchies and Stieltjes Constants]{Von Mangoldt Hierarchies and Stieltjes Constants in Explicit Formulas}

\author{Student Name}
\address{Department of Mathematics, [Your University], [Your City, State, ZIP]}
\email{student@email.edu}

\author{David England (Advisor)}
\address{Independent Researcher, Collinwood, TN, USA}
\email{contact@email.com}

\date{October 2025}

\keywords{Riemann zeta function, Stieltjes constants, von Mangoldt function, explicit formula, analytic number theory}
\subjclass[2020]{11M06, 11M26, 11N05}

\begin{document}

\begin{abstract}
We develop higher-order generalizations of the von Mangoldt function arising from derivatives of the logarithmic derivative of the Riemann zeta function. These \emph{von Mangoldt hierarchies}~$\Lambda_k(n)$ admit Dirichlet series
\[
(-1)^k \frac{d^k}{ds^k}\!\Big(-\frac{\zeta'(s)}{\zeta(s)}\Big)
= \sum_{n\ge1} \frac{\Lambda_k(n)}{n^s},
\]
and satisfy explicit prime-power formulas $\Lambda_k(p^r)=(\log p)^{k+1}r^k$.
We derive residue expansions for
\[
\Psi_k(x)=\sum_{n\le x}\Lambda_k(n),
\]
obtaining main terms involving Stieltjes constants~$\gamma_m$ and polynomial corrections in~$\log x$.
We present both conditional (under the Riemann Hypothesis) and unconditional estimates for the error terms, and discuss the cumulant interpretation of Stieltjes constants and their relation to the prime number distribution.
\end{abstract}

\maketitle
\tableofcontents

\section{Introduction}

The classical von Mangoldt function~$\Lambda(n)$ encodes prime powers via
\[
-\frac{\zeta'(s)}{\zeta(s)}=\sum_{n\ge1}\frac{\Lambda(n)}{n^s},\qquad \Re(s)>1.
\]
Its partial sums
\[
\Psi(x)=\sum_{n\le x}\Lambda(n)
\]
are central to the explicit formula linking prime distribution with zeros of the Riemann zeta function.

Differentiating the logarithmic derivative of~$\zeta(s)$ introduces powers of~$\log n$:
\[
\frac{d^k}{ds^k} n^{-s}=(-\log n)^k n^{-s}.
\]
Thus, successive derivatives yield \emph{weighted} von Mangoldt functions emphasizing logarithmic moments of prime powers.
These ``von Mangoldt hierarchies'' form a natural bridge between zero distributions, zeta derivatives, and the Laurent expansion coefficients of~$\zeta(s)$.

The Stieltjes constants~$\gamma_m$, defined by
\[
\zeta(1+u)=\frac{1}{u}+\sum_{m\ge0}\frac{(-1)^m\gamma_m}{m!}u^m,
\]
arise as the finite parts of~$\zeta$ near its pole at~$s=1$. They enter residue computations for~$\Psi_k(x)$, producing a hierarchy connecting the analytic structure at~$s=1$ with logarithmic weighting in the prime domain.

\section{Preliminaries}

\subsection{Definition of the von Mangoldt hierarchy}

For integers $k\ge0$, define
\[
\Lambda_k(n)\quad\text{by}\quad
(-1)^k\frac{d^k}{ds^k}\!\Big(-\frac{\zeta'(s)}{\zeta(s)}\Big)
= \sum_{n\ge1}\frac{\Lambda_k(n)}{n^s}.
\]
Differentiating the Euler product
\[
-\frac{\zeta'(s)}{\zeta(s)} = \sum_{p}\frac{\log p}{p^s-1},
\]
yields the explicit prime-power formula:
\[
\boxed{\Lambda_k(p^r) = (\log p)^{k+1}r^k, \quad r\ge1,}
\]
and $\Lambda_k(n)=0$ otherwise.

\subsection{Dirichlet series relations}

Note that $\Lambda_0(n)=\Lambda(n)$, and in general
\[
\Lambda_k(n) = (\log n)^k \, (\Lambda * \log^{*k})(n),
\]
where $*$ denotes Dirichlet convolution and $\log^{*k}$ is the $k$-fold convolution power of the logarithm sequence.

\subsection{Stieltjes constants}

The Stieltjes constants admit the limit representation
\[
\gamma_m = \lim_{N\to\infty}\Bigg(\sum_{n=1}^N\frac{(\log n)^m}{n}
- \frac{(\log N)^{m+1}}{m+1}\Bigg),
\]
and their exponential generating function is
\[
G(t)=\sum_{m\ge0}\frac{\gamma_m}{m!}t^m=\zeta(1-t)+\frac{1}{t}.
\]

\section{Main Theorem: Conditional Explicit Formula}

\begin{theorem}\label{thm:main}
For $k\ge0$, define
\[
\Psi_k(x)=\sum_{n\le x}\Lambda_k(n).
\]
Then for any $c>1$,
\[
\Psi_k(x)=\frac{1}{2\pi i}\int_{c-i\infty}^{c+i\infty}
(-1)^k\frac{d^k}{ds^k}\!\Big(-\frac{\zeta'(s)}{\zeta(s)}\Big)\frac{x^s}{s}\,ds.
\]
Shifting the contour leftward and summing residues gives
\[
\Psi_k(x)
= x(\log x)^k
-\sum_{m=0}^{k}\binom{k}{m}\gamma_m\,x(\log x)^{k-m}
+\sum_{\rho}x^{\rho}P_k(\rho,\log x)
+O(1),
\]
where the sum runs over nontrivial zeros $\rho$ of $\zeta(s)$, and $P_k$ is an explicit polynomial of degree~$k$ with coefficients depending on $\rho$ and $\gamma_m$.

Under the Riemann Hypothesis,
\[
\Psi_k(x) = x(\log x)^k + O\!\big(x^{1/2}\log^{k+2}x\big).
\]
\end{theorem}

\begin{proof}[Sketch of Proof]
Differentiating under the integral sign and applying Cauchy’s theorem isolates the residue at $s=1$, whose Laurent coefficients involve the Stieltjes constants~$\gamma_m$. Nontrivial zeros~$\rho$ contribute oscillatory secondary terms $x^{\rho}$ multiplied by polynomials in~$\log x$.
The resulting error term follows from standard zero-free region bounds or from the Riemann Hypothesis.
\end{proof}

\section{Computational Verification}

\subsection{Numerical objectives}
To verify the first coefficients of the residue expansion and observe the oscillatory zero terms, we:
\begin{enumerate}
    \item Compute $\gamma_m$ via high-precision zeta evaluation near $s=1$.
    \item Evaluate truncated sums $\Psi_k(x)$ for $k=0,1,2$ up to $x\le10^6$.
    \item Compare with the main term
    \[
    M_k(x)=x(\log x)^k-\sum_{m=0}^{k}\binom{k}{m}\gamma_m\,x(\log x)^{k-m}.
    \]
    \item Plot $\Psi_k(x)-M_k(x)$ to reveal oscillations corresponding to the first few zeta zeros.
\end{enumerate}

\subsection{Expected results}
The residual differences should oscillate with frequencies matching $\Im(\rho)$, confirming that higher derivatives of $\zeta$ emphasize deeper zero correlations and higher ``log moments'' of the prime structure.

\section{Discussion and Future Work}

\subsection{Cumulant interpretation}

Viewing $\gamma_m$ as cumulants of a hypothetical ``logarithmic prime distribution'' gives, via Bell polynomials,
\[
\mu_n = B_n(\gamma_1,\gamma_2,\dots,\gamma_n),
\]
formal ``moments'' related to sums of $\Lambda_k(n)$ under logarithmic weighting.  Clarifying this probabilistic model remains open.

\subsection{Open problems}

\begin{itemize}
    \item Uniform bounds for $\Psi_k(x)$ as $k\to\infty$.
    \item Asymptotics of $\gamma_m$ with explicit error terms (beyond Knessl–Coffey).
    \item Exact function-field analogues for $\Lambda_k$ in $\mathbb{F}_q[T]$.
    \item Possible spectral or trace interpretations of $\Lambda_k$ hierarchies.
\end{itemize}

\section*{Acknowledgements}
The authors thank [Department or Research Group] for support. The second author acknowledges stimulating discussions with colleagues working on prime distributions and Stieltjes asymptotics.

\begin{thebibliography}{99}

\bibitem{Titchmarsh}
E.~C. Titchmarsh, \emph{The Theory of the Riemann Zeta-Function}, 2nd~ed., Oxford Univ. Press, 1986.

\bibitem{Davenport}
H.~Davenport, \emph{Multiplicative Number Theory}, 3rd~ed., Springer GT, 2000.

\bibitem{IwaniecKowalski}
H.~Iwaniec and E.~Kowalski, \emph{Analytic Number Theory}, AMS Colloquium Publications, 2004.

\bibitem{KnesslCoffey}
C.~Knessl and M.~Coffey, \emph{An effective asymptotic formula for the Stieltjes constants}, Math. Comp.~80 (2011), 379--386.

\bibitem{GranvilleNotes}
A.~Granville, \emph{Analytic Number Theory Lectures}, available online notes, various years.

\end{thebibliography}

\end{document}